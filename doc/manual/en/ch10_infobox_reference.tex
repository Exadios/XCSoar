\chapter{InfoBox Reference}\label{cha:infobox}
InfoBox data types are grouped into logical categories.

All InfoBoxes display their data in user-specified units.  Whenever the content 
is invalid, the displayed value will be '---' and the content is
greyed out.  This happens, for example, when no terrain elevation is found for 
the `Terrain Elevation' InfoBox, or in the same way for a derived InfoBox like 
`Height AGL'.

Some of the InfoBox contents are modifiable complex values like `MC setting', or `Wind'. Most of 
those values are now accessible through an InfoBox dialogue. It is a short cut to quickly 
change the most often accessed items. An InfoBox dialogue is opened by long press to the InfoBox 
(Touchscreen devices), or `Select' and `Enter' (PC, Altair).

In the following description of the InfoBox types, the first
title is as it appears in the InfoBox configuration dialogue box, the
second title is the label used in the InfoBox title.

\newcommand{\ibi}[3]{%
\jindent{
\begin{tabular}{r}
{\bf #1} \\
\infobox{{#2}} \\
\end{tabular}}{#3}
}
\newcommand{\ibig}[4]{%
\jindent{
\begin{tabular}{r}
{\bf #1} \\
\infobox{{#2}} \\
\includegraphics[width=3.5cm,keepaspectratio='true']{#4} \\
\end{tabular}}{#3}
}


%%%%%%%%%%%
\section{Altitude}

\ibig{Altitude GPS}{Alt GPS}{This is the altitude above mean sea level reported by the
GPS. (Touch-screen/PC only) In simulation mode this value is adjustable with the
up/down arrow keys. The right/left arrow keys also cause the glider to turn.\footnotemark[1]}
{figures/simulator-keys.png}
\ibi{Barometric altitude}{Alt Baro}{This is the barometric altitude obtained from a
device equipped with pressure sensor.\footnotemark}
\ibi{Altitude (Auto)}{Alt $<$auto$>$}{This is the barometric altitude obtained from a 
device equipped with a pressure sensor or the GPS altitude if the barometric altitude 
is not available.}
\ibi{Height AGL}{H AGL}{This is the navigation altitude minus the terrain elevation 
obtained from the terrain file. The value is coloured red when the glider is
below the terrain safety clearance height.\footnotemark[1]}  
\ibi{Terrain elevation}{Terr Elev}{This is the elevation of the terrain above mean
sea level obtained from the terrain file at the current GPS location.}
\ibi{Height above take-off}{H T/O}{Height based on an automatic take-off reference 
elevation (like a QFE reference).\footnotemark[1]}
\ibi{Flight level}{Flight Level}{Pressure Altitude given as Flight Level. 
Only available if barometric altitude available and correct QNH set.\footnotemark[1]}
\ibi{Barogram}{Barogram}{Trace of altitude during flight.}

\footnotetext[1]{In simulator mode an additional dialogue is available to change the value 
of the InfoBox.}


%%%%%%%%%%%
\section{Aircraft state}

\ibi{Speed ground}{V GND}{Ground speed measured by the GPS. If this InfoBox is
active in simulation mode, pressing the up and down arrows adjusts the speed, 
left and right turns the glider.}
\ibi{Track}{Track}{Magnetic track reported by the GPS. (Touch-screen/PC only) If
this InfoBox is active in simulation mode, pressing the up and down arrows
adjusts the track.}
\ibi{Airspeed IAS}{V IAS}{Indicated Airspeed reported by a supported external
intelligent vario.}
\ibi{G load}{G}{Magnitude of G loading reported by a supported external
intelligent vario. This value is negative for pitch-down manoeuvres.}
\ibi{Bearing Difference}{Brng. D}{The difference between the glider's track
bearing, to the bearing of the next waypoint, or for AAT tasks, to the bearing
to the target within the AAT sector. GPS navigation is based on the track
bearing across the ground, and this track bearing may differ from the glider's
heading when there is wind present. Chevrons point to the direction the glider
needs to alter course to correct the bearing difference, that is, so that the
glider's course made good is pointing directly at the next waypoint.  This
bearing takes into account the curvature of the Earth.}
\ibi{Airspeed TAS}{V TAS}{True Airspeed reported by a supported external 
intelligent vario.}
\ibi{Attitude indicator}{Horizon}{Attitude indicator (artificial horizon) display 
calculated from flightpath, supplemented with acceleration and variometer data if 
available.}


%%%%%%%%%%%
\section{Glide ratio}

\ibi{GR instantaneous}{GR Inst}{Instantaneous glide ratio over ground, given 
by the ground
speed divided by the vertical speed (GPS speed) over the last 20 seconds. 
Negative values indicate climbing cruise. If the vertical speed is close to
zero, the displayed value is '---'.}
\ibi{GR cruise}{GR Cruise}{The distance from the top of the last thermal,
divided by the altitude lost since the top of the last thermal. Negative values
indicate climbing cruise (height gain since leaving the last thermal). If the
vertical speed is close to zero, the displayed value is '---'.}
\ibi{Final GR}{Fin GR}{The required glide ratio over ground to finish the task, given 
by the distance to go divided by the height required to arrive at the safety 
arrival height. This is no adjusted total energy possible.} 
\ibi{Next GR}{WP GR}{The required glide ratio over ground to reach the next waypoint,
given by the distance to next waypoint divided by the height required to arrive
at the safety arrival height.  Negative values indicate a climb is necessary
to reach the waypoint.  If the height required is close to zero, the displayed
value is '---'.}
\ibi{L/D vario}{L/D Vario}{Instantaneous lift/drag ratio, given by the indicated
airspeed divided by the total energy vertical speed, when connected to an
intelligent variometer.  Negative values indicate climbing cruise. If the total
energy vario speed is close to zero, the displayed value is '---'.}
\ibi{GR average}{GR Avg}{The distance made in the configured period of time ,
divided by the altitude lost since then. Negative values are shown as 
\^{ }\^{ }\^{ } and indicate climbing cruise (height gain). Over 200 of GR the
value is shown as +++ . You can configure the period of averaging.
Suggested values are 60, 90 or 120. Lower values will be closed to 
GR inst., and higher values will be closed to GR cruise. Notice that the distance 
is \textit{not} the straight line between your old and current position, it's exactly the 
distance you have made even in a zigzag glide. This value is not calculated while 
circling.}

%%%%%%%%%%%
\section{Variometer}

\ibi{Last Thermal Average}{TL Avg}{Total altitude gain/loss in the last thermal
divided by the time spent circling.} 
\ibi{Last thermal gain}{TL Gain}{Total altitude gain/loss in the last thermal.}
\ibi{Last thermal time}{TL Time}{Time spent circling in the last thermal.}
\ibi{Thermal climb, last 30 s}{TC 30s}{A 30 second rolling average climb rate based
of the reported GPS altitude, or vario if available.}
\ibi{Thermal average}{TC Avg}{Altitude gained/lost in the current thermal,
divided by time spent thermalling.}
\ibi{Thermal gain}{TC Gain}{The altitude gained/lost in the current thermal.}
\ibi{Vario }{Vario}{Instantaneous vertical speed, as reported by the GPS, or the
intelligent vario total energy vario value if connected to one.}
\ibi{Netto vario}{Netto}{Instantaneous vertical speed of air-mass, equal to
vario value less the glider's estimated sink rate. Best used if airspeed,
accelerometers and vario are connected, otherwise calculations are based on GPS
measurements and wind estimates.}
\ibi{Vario trace}{Vario Trace}{Trace of vertical speed, as reported by the GPS, 
or the intelligent vario total energy vario value if connected to one.}
\ibi{Netto vario trace}{Netto Trace}{Trace of vertical speed of air-mass, equal 
to vario value less the glider's estimated sink rate.}
\ibi{Thermal climb trace}{TC Trace}{Trace of average climb rate each turn in 
circling, based of the reported GPS altitude, or vario if available.}
\ibi{Thermal average over all}{T Avg}{Time-average climb rate in all thermals.}
\ibi{Climb band}{Climb Band}{Graph of average circling climb rate (horizontal 
axis) as a function of altitude (vertical axis).}
\ibi{Thermal assistant}{Thermal}{A circular thermal assistant that shows the 
lift distribution over each part of the circle.}

%%%%%%%%%%%
\section{Atmosphere}

\ibig{Wind arrow}{Wind}{Wind vector estimated by XCSoar. Manual adjustment is possible 
with the connected InfoBox dialogue. Pressing the up/down cursor keys to cycle through 
settings, adjust the values with left/right cursor keys.}
{figures/infobox-dialog-wind1.png}
\ibi{Wind bearing}{Wind Brng}{Wind bearing estimated by XCSoar. Adjustable in the same 
manner as Wind arrow.}
\ibi{Wind speed}{Wind V}{Wind speed estimated by XCSoar. Adjustable in the same 
manner as Wind arrow.}
\ibi{Head wind component}{Head Wind}{The current head wind component. Head wind 
is calculated from TAS and GPS ground speed if airspeed is available from 
external device. Otherwise the estimated wind is used for the calculation.}
\ibi{Head wind component (simplified)}{Head Wind *}{The current head wind component. 
The simplified head wind is calculated by subtracting GPS ground speed from the TAS if 
airspeed is available from external device.}
\ibi{Outside air temperature}{OAT}{Outside air temperature measured by a probe
if supported by a connected intelligent variometer.}
\ibi{Relative humidity}{Rel Hum}{Relative humidity of the air in percent as
measured by a probe if supported by a connected intelligent variometer.}
\ibi{Forecast temperature}{Max Temp}{Forecast temperature of the ground at the
home airfield, used in estimating convection height and cloud base in
conjunction with outside air temperature and relative humidity probe. 
(Touch-screen/PC only) Pressing the up/down cursor keys adjusts this forecast
temperature.}

%%%%%%%%%%%
\section{MacCready}

\ibi{MacCready Setting}{MC $<$mode$>$}{The current MacCready setting, the current 
MacCready mode (manual or auto), and the recommended speed-to-fly. 
(Touch-screen/PC only) Also used
to adjust the MacCready setting if the InfoBox is active, by using the up/down
cursor keys. Pressing the enter cursor key toggles `Auto MacCready' mode. 
An InfoBox dialogue is available}
\ibi{Speed MacCready}{V MC}{The MacCready speed-to-fly for optimal flight to the
next waypoint. In cruise flight mode, this speed-to-fly is calculated for
maintaining altitude. In final glide mode, this speed-to-fly is calculated for
descent.}
\ibi{Percentage climb}{\% Climb}{Percentage of time spent in climb mode. These
statistics are reset upon starting the task.}
\ibi{Speed dolphin}{V opt.}{The instantaneous MacCready speed-to-fly, making use
of netto vario calculations to determine dolphin cruise speed in the glider's
current bearing. In cruise flight mode, this speed-to-fly is calculated for
maintaining altitude. In final glide mode, this speed-to-fly is calculated for
descent. In climb mode, this switches to the speed for minimum sink at the
current load factor (if an accelerometer is connected). When `Block' mode speed to
fly is selected, this InfoBox displays the MacCready speed.}
\ibi{Thermal next leg equivalent}{T Next Leg}{The thermal rate of climb on next 
leg which is equivalent to a thermal equal to the MacCready setting on current leg.}
\ibi{Task cruise efficiency}{Cruise Eff}{Efficiency of cruise. 100 indicates perfect 
MacCready performance. This value estimates your cruise efficiency according to the 
current flight history with the set MC value.  Calculation begins after task is started.}

%%%%%%%%%%%
\section{Navigation}

\ibi{Next Bearing}{Bearing}{True bearing of the next waypoint. For AAT tasks, this
is the true bearing to the target within the AAT sector.}
\ibi{Next radial}{Radial}{True bearing from the next waypoint to your position.}
\ibi{Next distance}{WP Dist}{The distance to the currently selected waypoint.
For AAT tasks, this is the distance to the target within the AAT sector.}
\ibi{Next altitude difference}{WP AltD}{Arrival altitude at the next waypoint
relative to the safety arrival height. For AAT tasks, the target within the AAT 
sector is used.}
\ibi{Next MC0 altitude difference}{WP MC0 AltD}{Arrival altitude at the next 
waypoint with MC 0 setting relative to the safety arrival height. For AAT tasks, 
the target within the AAT sector is used.}   
\ibi{Next altitude arrival}{WP AltA}{Absolute arrival altitude at the next waypoint 
in final glide. For AAT tasks, the target within the AAT sector is used.}
\ibi{Next altitude required}{WP AltR}{Altitude required to reach the next turn
point. For AAT tasks, the target within the AAT sector is used.}
\ibi{Final altitude difference}{Fin AltD}{Arrival altitude at the final task
turn point relative to the safety arrival height.}
\ibi{Final altitude required}{Fin AltR}{Additional altitude required to finish the task.}
\ibi{Final distance}{Final Dist}{Distance to finish around remaining turn points.}
\ibi{Distance home}{Home Dist}{Distance to the home waypoint (if defined).}


%%%%%%%%%%%
\section{Competition and assigned area tasks}
\ibi{Speed task average}{V Task Avg}{Average cross country speed while on
current task, not compensated for altitude.}
\ibi{Speed task instantaneous}{V Task Inst}{Instantaneous cross country speed
while on current task, compensated for altitude.  Equivalent to instantaneous 
Pirker cross-country speed.}
\ibi{Speed task achieved}{V Task Ach}{Achieved cross country speed while on
current task, compensated for altitude.  Equivalent to Pirker cross-country 
speed remaining.}
\ibi{AAT time}{AAT Time}{ `Assigned Area Task' time remaining. Goes red when time 
remaining has expired.}
\ibi{AAT delta time}{AAT dT}{Difference between estimated task time and 
AAT minimum time. Coloured red if negative (expected arrival too early), or 
blue if in sector and can turn now with estimated arrival time greater than 
AAT time plus 5 minutes.}
\ibi{AAT max. distance}{AAT Dmax}{ `Assigned Area Task' maximum distance possible for
remainder of task.}
\ibi{AAT min. distance}{AA Dmin}{ `Assigned Area Task' minimum distance possible for
remainder of task.}
\ibi{AAT speed max. distance}{AAT Vmax}{ `Assigned Area Task' average speed achievable if
flying maximum possible distance remaining in minimum AAT time.}
\ibi{AAT speed min. distance}{AAT Vmin}{ `Assigned Area Task' average speed achievable if
flying minimum possible distance remaining in minimum AAT time.}
\ibi{AAT distance around target}{AAT Dtgt}{`Assigned Area Task' distance around target points
for remainder of task.}
\ibi{AAT speed around target}{AAT Vtgt}{`Assigned Area Task' average speed achievable around
target points remaining in minimum AAT time.}
\ibi{On-Line Contest distance}{OLC}{Instantaneous evaluation of the flown
distance according to the configured Online-Contest rule set.}
\ibi{Task progress}{Progress}{Clock-like display of distance remaining along 
task, showing achieved task points.}
\ibi{Start open/close countdown}{Start open}{Shows the time left until the start point 
opens or closes.}
\ibi{Start open/close countdown at reaching}{Start reach}{Shows the time left until the 
start point opens or closes, compared to the calculated time to reach it.}
    
    
%%%%%%%%%%%
\section{Waypoint}

\ibi{Next waypoint}{Next WP}{The name of the currently selected turn point. When
this InfoBox is active, using the up/down cursor keys selects the next/previous
waypoint in the task. (Touch-screen/PC only) Pressing the enter cursor key brings
up the waypoint details.}
\ibi{Flight Duration}{Flt Duration}{Time elapsed since take-off was detected.}
\ibi{Time local}{Time loc}{GPS time expressed in local time zone.}
\ibi{Time UTC}{Time UTC}{GPS time expressed in UTC.}
\ibi{Task time to go}{Fin ETE}{Estimated time required to complete task,
assuming performance of ideal MacCready cruise/climb cycle.}
\ibi{Task time to go (ground speed)}{Fin ETE VMG}{Estimated time required to 
complete task, assuming current ground speed is maintained.}
\ibi{Next time to go}{WP ETE}{Estimated time required to reach next waypoint,
assuming performance of ideal MacCready cruise/climb cycle.}
\ibi{Next time to go (ground speed)}{WP ETE VMG}{Estimated time required to 
reach next waypoint, assuming current ground speed is maintained.}
\ibi{Task arrival time}{Fin ETA}{Estimated arrival local time at task
completion, assuming performance of ideal MacCready cruise/climb cycle.}
\ibi{Next arrival time}{WP ETA}{Estimated arrival local time at next waypoint,
assuming performance of ideal MacCready cruise/climb cycle.}
\ibi{Task req. total height trend}{RH Trend}{Trend (or neg. of the variation) of
the total required height to complete the task.}
\ibi{Time under max. start height}{Start Height}{The contiguous period the ship 
has been below the task start max. height.}


%%%%%%%%%%%
\section{Team code}

\ibi{Team code}{Team Code}{The current Team code for this aircraft. Use this
to report to other team members.  The last team aircraft code entered is 
displayed underneath.}
\ibi{Team bearing}{Team Brng}{The bearing to the team aircraft location at the
last team code report.}
\ibi{Team bearing difference}{Team BrngD}{The relative bearing to the team aircraft
location at the last reported team code.}
\ibi{Team range}{Team Dist}{The range to the team aircraft location at the last
reported team code.}

%%%%%%%%%%%
\section{Device status}

\ibi{Battery voltage/percent}{Battery}{Displays percentage of device battery remaining
(where applicable) and status/voltage of external power supply.}
\ibi{CPU load}{CPU}{CPU load consumed by XCSoar averaged over 5 seconds.}
\ibi{Free RAM}{Free RAM}{Free RAM as reported by the operating system.}

%%%%%%%%%%%
\section{Alternates}

\ibi{Alternate 1}{Altn 1}{Displays name and bearing to the best alternate
landing location.}
\ibi{Alternate 2}{Altn 2}{Displays name and bearing to the second alternate
landing location.}
\ibi{Alternate 1 GR}{Altn1 GR}{Geometric gradient to the arrival height above
the best alternate. This is not adjusted for total energy.}

%%%%%%%%%%%
\section{Obstacles}

\ibi{Nearest airspace horizontal}{Near AS H}{The horizontal distance to the 
nearest airspace.}
\ibi{Nearest airspace vertical}{Near AS V}{The vertical distance to the nearest 
airspace.  A positive value means the airspace is above you, and negative means 
the airspace is below you.}
\ibi{Terrain collision}{Terr Coll}{The distance to the next terrain collision 
along the current task leg. At this location, the altitude will be below the 
configured terrain clearance altitude.}


