\chapter{Installation}\label{cha:installation}

To run XCSoar, you need to obtain the following:

\begin{itemize}
\item a device to run XCSoar on
\item XCSoar
\item a GPS receiver
\item a waypoint file
\item an airspace file (optional)
\item a map file (optional)
\end{itemize}

\section{Compatibility}

\subsection*{Devices for running XCSoar}

XCSoar runs on the following platforms:

\begin{itemize}
\item mobile phones and tablets with Android 1.6 or newer \\
  Example: Dell Streak, Samsung Galaxy S II, HTC Desire HD,
  Motorola Xoom
\item PDAs with Pocket PC 2000, 2002, 2003 \\
  Example: iPaq 3800, iPaq 3900
\item PDAs with Windows Mobile \\
  Example: iPaq hx4700, Dell Axim x51v
\item PNAs with Windows CE 3.0 or newer \\
  Example: HP314, Mio400
\item Triadis Altair
\item LX MiniMap
\item Windows 2000 or newer
\item Linux
\end{itemize}

\subsection*{GPS, Logger, Vario}

XCSoar is compatible with any GPS emitting NMEA data.  Most modern
Android devices have a built-in GPS, but sometimes it is favorable to
connect to an external device:

\begin{itemize}
\item an airspeed indicator allows quick and exact wind estimates
  without circling
\item a vario improves the thermal assistant
\item a task can be declared to an IGC logger, and after landing, the
  flight log can be downloaded
\item some varios allow synchronising the MacCready setting with
  XCSoar
\end{itemize}

\subsection*{Supported external devices and features}
\label{sec:supported-varios}

\noindent\makebox[\textwidth]{%
\begin{tabularx}{1.2\textwidth}{l|cccc}

NMEA & Feature \\
Device & Airspeed & Vario & Declaration & Download \\
\hline

Borgelt B50 & $\surd$ \\

CAI 302 & $\surd$ & $\surd$ & $\surd$ & $\surd$ \\

CAI GPS Nav \\

Condor & $\surd$ \\
\hline

Digifly Leonardo & $\surd$ & $\surd$ \\

EW Logger &&& $\surd$ \\

EW microRecorder &&& $\surd$ \\

Flarm &&& $\surd$ & $\surd$ \\
\hline

Flymaster F1 && $\surd$ \\

Flytec 5030 & $\surd$ & $\surd$ \\

ILEC SN10 & & $\surd$ \\

IMI ERIXX &&& $\surd$ & $\surd$ \\
\hline

LX20, Colibri &&& $\surd$ & $\surd$ \\

LX 5000, 7000 & $\surd$ & $\surd$ & $\surd$ \\

LXNAV V7 & $\surd$ & $\surd$ \\

PosiGraph &&& $\surd$ \\

Triadis Altair &&& $\surd$ \\
\hline

Triadis Vega & $\surd$ & $\surd$ \\

Volkslogger &&& $\surd$ \\

Westerboer VW1150 & $\surd$ & $\surd$ \\

Westerboer VW921, 922 & $\surd$ & $\surd$ \\
\hline

Zander / SDI & $\surd$ & $\surd$ \\

\end{tabularx}}


While most Windows CE based devices have a serial port, such legacy
hardware is not present in modern Android devices.  Those can either
use Bluetooth or the Android IOIO board.  To use Bluetooth, you need
to connect the external device to a Bluetooth-to-Serial adapter, such
as the K6-Bt or the Glidertools VFBT-1.


\section{Software installation}

The software is available as a free download from the XCSoar website
~\xcsoarwebsite{}.  This section describes which file should be
downloaded, and how to install it.

\subsection*{On Android}

Obtain XCSoar from Google's Android market, or install the \verb|apk|
file manually.  Copy the data files on the SD card in the directory
\verb|XCSoarData|.

\subsection*{On a PDA (Windows Mobile, PocketPC)}

Choose one of the targets:

\begin{description}
\item[\texttt{PPC2000}] Pocket PC 2000/2002, Windows CE 3.0
\item[\texttt{PPC2003}] Pocket PC 2003, Windows CE 4.0
\item[\texttt{WM5}] Windows Mobile 5 or newer
\item[\texttt{WM5X}] Windows Mobile 5 or newer with XScale CPU or
  better (e.g. hx4700)
\end{description}

Download the program file \verb|XCSoar.exe| to a SD card.  You can
launch it with the File Explorer.
\sketch{figures/XCS_Today.png}
Another method to install XCSoar on a PDA is the CAB file.  Download
it to the SD card.  Use the File Explorer to install it.  After the
installation, the XCSoar `FLY' and `SIM' launcher icons will be
visible on the Today screen.


\subsection*{On a PNA (Windows CE)}

Download the program file \verb|XCSoar.exe| (target ``WM5'') to a SD
card.  You can launch it with the File Explorer.

\subsection*{On a Windows PC}

Download the program file \verb|XCSoar.exe| (target ``PC'') to your
hard disk.

\subsection*{On Unix/Linux}

The file downloaded is \verb|xcsoar_XXX.deb|, where \verb|XXX| includes
the version number and platform, e.g. \verb|xcsoar_6.0.4_i386.deb|.
The is a Debian package and can be installed using 
\begin{center}
\verb|sudo dpkg -i xcsoar_XXX.deb|.
\end{center}
Use \verb|dpkg-query -L xcsoar| to see where the executable and 
other files are installed,
Additional data files must be placed in the directory
\verb|~/.xcsoar/XCSoarData/|.
If \verb|~/.xcsoar| does not exist, it will be created the first time
that \verb|xcsoar| is run.


\section{Data files}

To be able to use XCSoar's advanced features, additional data files, such as
terrain, topography, special use airspace, waypoints etc.\ are needed. The files
that can be used with XCSoar are described in Chapter~\ref{cha:data-files}.

All data files should be copied into the directory
\texttt{XCSoarData}.  This directory must be in a specific place
so that XCSoar knows where to look for data files:

\begin{description}
\item[Windows PC]
\texttt{XCSoarData} is in your personal folder (``\texttt{My
Documents}'')
\item[Windows Mobile PDA/PNA]
\texttt{XCSoarData} is on the SD card.  If there is no SD card, then
XCSoar looks for it in \texttt{My Documents}.
\item[Unix/Linux]
The directory is called \verb|.xcsoar| in the user's home directory.
\item[Android Devices]
\texttt{XCSoarData} is on the SD card.
\item[Altair]
If XCSoarData exists on an USB drive, that one is used, otherwise the
internal storage is used.
\end{description}


XCSoar will generate a number of additional files at run time.  These
will be placed in the  \texttt{XCSoarData} directory (Windows PC and 
Windows Mobile devices), or the \texttt{.xcsoar} directory (Unix/Linux
PC).  At first run, XCSoar will create the files 
\texttt{Default.tsk} (Default Task),  \texttt{xcsoar-registry.prf} 
(configuration settings), \newline
\texttt{xcsoar-startup.log} (log of the startup progress), 
plus three directories: \texttt{cache},
\texttt{config} and \texttt{logs}.  Additional files may be
created/modified while XCSoar is running, such as task files
(\texttt{*.tsk}) and flight logs.


\section{Running XCSoar}
%\subsection*{Fly and simulator modes}

Two modes are available inside the XCSoar application: 
\begin{description}
\item[FLY] This mode is used when actually flying.  The simulator is 
  disabled and serial communications are active. 
\item[SIM] This starts XCSoar in simulator mode, no serial communications
  are attempted.
\end{description}

\subsection*{Altair version}
XCSoar starts up automatically when Altair is powered on.
The PWR/ESC button (top left) has multiple functions:
\begin{description}
\item[Powering on]  Press and hold the PWR/ESC button for one second.
  The LED in the button will light up, and XCSoar will start after
  Altair has booted.
\item[Powering off]  Press and hold the PWR/ESC button for 3 seconds.
  Altair will switch off.
\item[Escape] Pressing the PWR/ESC button quickly acts as an
Escape key, typically used to close dialog pages or as a cancel function.
\end{description}

The Altair version of XCSoar does not include a simulator mode.

\subsection*{XCSoar PC version}
The program can be run by opening the explorer window, finding the directory
that has the XCSoar.exe executable, and double clicking on that program file.

The program command line options allows the screen orientation of
the display to be defined:
\begin{description}
\item[-portrait] The screen is 480 pixels wide, 640 pixels high.
\item[-square] The screen is 480 pixels wide, 480 pixels high.
\item[-landscape] The screen is 640 pixels wide, 480 pixels high. This is the
usual setting. If you don't specify this parameter the landscape version will be
loaded automatically.
\item[-small] Draws the screen at half size.  This is useful for using XCSoar in
 conjunction with flight simulators e.g.\ Condor.
\end{description}
To change the screen orientation, it is convenient to create a shortcut to the
program, then right click on the shortcut icon and click on ``Shortcut''. 
In the field ``Target'' add one of the desired options listed above.

\subsection*{XCSoar Unix/Linux PC version}
Run \verb|xcsoar| from a command line, or create a shortcut on the
desktop.  The location of the executable file may be found using
\verb|which xcsoar|.  Only landscape mode is  supported for now.

\subsection*{Loading data files}
The first time that XCSoar is run, it does not automatically load the 
data files that you placed in the \verb|XCSoarData| directory.  
To tell XCSoar which files to load, double click/tap the map (the large,
blank white part with the glider symbol in the center),
choose the menu \bmenu{Config} (click/tap it twice), then select
\bmenu{System Setup}.  The System Setup screen should be displayed:
\sketch{figures/config-basic.png}
The first page allows you to choose the map, 
waypoint and airspace files, by clicking/tapping on the text boxes.
Many other features of XCSoar may be configured with
\bmenu{System Setup}. These are described in detail in Chapter
\ref{cha:configuration}.
Once completed, XCSoar reloads those files; from now on, the data files
will be loaded automatically at run time.

\subsection*{Start-up and user profiles}
When XCSoar starts up, it will check for existing profiles. If multiple
profiles are detected it will displays a small window asking you which profile
to load. To proceed, choose the desired profile and press Enter. If no
profile is chosen the settings from the last session are loaded again. Profiles
can be useful for example in the following cases:
\begin{itemize}
\item Different pilots
\item Competition versus casual flying
\item Flying in different locations
\end{itemize}


\subsection*{SIM mode}
The simulator contains a simple interface allowing the user to fly
the glider about.  On the map screen, clicking/touching the glider symbol
(with touchscreen or mouse) and dragging 
causes the glider to move in the direction of the drag, the
speed being proportional to the length of the drag.  

In the PC version and for embedded devices with buttons, the aircraft
speed, height and direction may be changed using the \InfoBox es.
These features are not available for touchscreen devices.
The aircraft altitude can be adjusted by selecting the GPS altitude
{\InfoBox} (marked \infobox{H GPS}), and pressing the up or down key.
The airspeed  can be adjusted by selecting the ground speed
{\InfoBox} (marked \infobox{V Gnd}), and pressing the up or down key.
The glider's track  can be adjusted by selecting the track
{\InfoBox} (marked \infobox{Track}), and pressing the up or down key.
With either of the \InfoBox es \infobox{H GPS} or \infobox{V Gnd})
selected, the glider's direction may be changed using the left/right keys.

Other controls, buttons and menus work the same as in FLY mode.


\subsection*{Splash screen}
When XCSoar starts up, shuts down, or loads large files, such as airspace,
waypoints, terrain, etc., a progress screen is presented while the data is being
loaded. This screen has a progress bar which indicates the data loading
activity, and a short line of text describing the action that is being performed.

This screen also displays the software version information.

\subsection*{Exiting the program}
For PDA and PC versions, XCSoar is shut down from the menu. The menu can be
opened by double-clicking on the map or the \InfoBox es.
\begin{quote}
\bmenu{QUIT}
\end{quote}

For PC versions, XCSoar can also be shut down by clicking the close icon
on the XCSoar window.

For Altair, XCSoar is shut down by holding the PWR button for two seconds or
more.

