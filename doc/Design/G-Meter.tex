% This file was converted to LaTeX by Writer2LaTeX ver. 0.5
% see http://www.hj-gym.dk/~hj/writer2latex for more info
\documentclass[a4paper]{report}
\usepackage[ascii]{inputenc}
\usepackage[T1]{fontenc}
\usepackage[english]{babel}
\usepackage{amsmath,amssymb,amsfonts,textcomp}
\usepackage{color}
\usepackage{array}
\usepackage{supertabular}
\usepackage{hhline}
\usepackage{hyperref}
\hypersetup{colorlinks=true, linkcolor=blue, citecolor=blue, filecolor=blue, pagecolor=blue, urlcolor=blue, pdftitle="Weather Radar Theory", pdfauthor=pfb, pdfsubject=, pdfkeywords=}
\usepackage[pdftex]{graphicx}
% Text styles
\newcommand\textstyleEmphasis[1]{\textit{#1}}
% Outline numbering
\setcounter{secnumdepth}{4}
\renewcommand\thesection{\arabic{chapter}.\arabic{section}}
\renewcommand\thesubsection{\arabic{chapter}.\arabic{section}.\arabic{subsection}}
\renewcommand\thesubsubsection{\arabic{chapter}.\arabic{section}.\arabic{subsection}.\arabic{subsubsection}}
\renewcommand\theparagraph{\arabic{chapter}.\arabic{section}.\arabic{subsection}.\arabic{subsubsection}.\arabic{paragraph}}
\makeatletter
\newcommand\arraybslash{\let\\\@arraycr}
\makeatother
% Figure numbering
\renewcommand{\thefigure}{\arabic{chapter}.\arabic{figure}}
\newcommand{\Section}[1]{\section{#1} \setcounter{figure}{1}}
% Footnote rule
\setlength{\skip\footins}{1.2mm}
\renewcommand\footnoterule{\vspace*{-0.18mm}\setlength\leftskip{0pt}\setlength\rightskip{0pt plus 1fil}\noindent\textcolor{black}{\rule{0.25\columnwidth}{0.18mm}}\vspace*{1.02mm}}
% Pages styles
\makeatletter
\newcommand\ps@NewChapter{
  \renewcommand\@oddhead{}
  \renewcommand\@evenhead{}
  \renewcommand\@oddfoot{}
  \renewcommand\@evenfoot{}
  \renewcommand\thepage{\arabic{page}}
}
\newcommand\ps@Standard{
  \renewcommand\@oddhead{}
  \renewcommand\@evenhead{}
  \renewcommand\@oddfoot{}
  \renewcommand\@evenfoot{}
  \renewcommand\thepage{\arabic{page}}
}
\newcommand\ps@FirstPage{
  \renewcommand\@oddhead{}
  \renewcommand\@evenhead{}
  \renewcommand\@oddfoot{}
  \renewcommand\@evenfoot{}
  \renewcommand\thepage{\arabic{page}}
}
\newcommand\ps@Contents{
  \renewcommand\@oddhead{}
  \renewcommand\@evenhead{}
  \renewcommand\@oddfoot{}
  \renewcommand\@evenfoot{}
  \renewcommand\thepage{\roman{page}}
}
\makeatother
\pagestyle{plain}
\setlength\tabcolsep{1mm}
\renewcommand\arraystretch{1.3}
\newcounter{Text}[section]
\numberwithin{equation}{chapter}
\renewcommand\theText{\thesection.\arabic{Text}}
\pagenumbering{arabic}
\renewcommand{\thepage}{\arabic{chapter}.\arabic{page}}

\title{G Meter Variometer Design}

\begin{document}
\maketitle
\date{January 2011}

\clearpage\setcounter{page}{1}
\thispagestyle{Contents}

\tableofcontents

\clearpage\setcounter{page}{1}
\thispagestyle{Contents}

\listoffigures


\clearpage\setcounter{page}{1}
\chapter[Introduction]{Introduction}

\section[Purpose]{Purpose}

The purpose of this design document is to describe an instrument that will provide the glider pilot with real time data that is compariable to the data presently provided by the traditional Variometer. However, this data will be derived in a completely different manner from the manner in which the traditional Variometer data is derived.

\bigskip

A glider climbs as a result of an acceleration normal to the Geod. By observing the normal acceleration the climb velocity may be determined.

\begin{equation}
v = \int{\left( a - g \right)} dt
\end{equation}

where $a$ is the observed acceleration normal to to the Geoid, $g$ is the acceleration due to gravity and $v$ is the velocity normal to the Geod.

\clearpage\setcounter{page}{1}
\chapter[Design]{Design}

In order to observe $a$ it is necessary to have an accelerometer vector which is aligned with the Geodetical normal. This requires a level stable platform. This platform will consist of a "strapdown" inertial platform.

\bigskip

Given the quality of avaliable inertial sensors (gyros, accelerometers) it is unrealistic to attempt to build a stand alone dead reckoning inertial platform. It will be necessary to couple the platform to a GPS solution of position ($\phi$, $\lambda$, $h$, $t$). This coupling will need to be fairly tight - time constants will need to be less than sixty seconds.

\bigskip

The necessary integration of the intertial platform and the GPS solution will be effected using a suitably designed Kalman Filter.

\appendix

\clearpage\setcounter{page}{1}
\clearpage\setcounter{page}{1}
\chapter[Kalman Filter]{Kalman Filter}

\section[The Discrete Kalman Filter Algorithm]{The Discrete Kalman Filter Algorithm}

The Kalman Filter Algorithm is implemented as a two step process. The first step is to generate an innovation at time $t_{k-1}$ by using the state of the system at time $t_{k - 1}$ and the model of the plant to produce the state at $t_k$. The second step is to introduce corrections to the innovation at time $t_k$ from observations made at $t_k$.

\bigskip

In the following the subscript notation $k|i$ donates the state at $t_k$ using knowledge from $t_i$. Further

$\hat{x}$= System state estimate

$u$ = Any control input that may be applied directly to the $\hat{x}$

$A$ = The matrix with maps $\hat{x}_{k - 1|k - 1}$ to $\hat{x}_{k|k - 1}$

$B$ = The matrix which maps $u$ to $\hat{x}$

$H$ = The matrix which maps $\hat{x}_{k|k - 1}$ to $z_{k|k}$

$P$ = Process covariance

$K$ = Kalman gain

$z$ = Observation

$Q$ = Process noise

$R$ = Observation noise

\subsubsection[Innovation]{Innovation}

\begin{equation}
\hat{x}_{k|{k-1}} = A\hat{x}_{{k-1}|{k-1}} + Bu_k
\end{equation}

\begin{equation}
P_{k|{k-1}} = AP_{{k-1}|{k-1}}A^T + Q
\end{equation}

\subsubsection[Observation]{Observation}

\begin{equation}
K_{k|k} = P_{k|{k-1}}H^T \left( HP_{k|{k-1}}H^T + R \right) ^{-1}
\end{equation}

\begin{equation}
\hat{x}_{k|k} = \hat{x}_{k|{k-1}} + K_{k|k} \left( z_{k|k} - H\hat{x}_{k|{k-1}} \right)
\end{equation}

\begin{equation}
P_{k|k} = \left( I - K_{k|k}H \right) P_{k|{k-1}}
\end{equation}

\end{document}
